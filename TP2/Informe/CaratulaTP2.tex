%% LyX 2.2.2 created this file.  For more info, see http://www.lyx.org/.
%% Do not edit unless you really know what you are doing.
\documentclass[a4paper,english]{article}
\usepackage[T1]{fontenc}

\makeatletter

%%%%%%%%%%%%%%%%%%%%%%%%%%%%%% LyX specific LaTeX commands.
\pdfpageheight\paperheight
\pdfpagewidth\paperwidth


%%%%%%%%%%%%%%%%%%%%%%%%%%%%%% User specified LaTeX commands.
\newcommand{\materia}[1]{\def\lamateria{#1}}
\newcommand{\trabajo}[1]{\def\eltrabajo{#1}}
\newcommand{\titulo}[1]{\def\eltitulo{#1}}
\newcommand{\fecha}[1]{\def\lafecha{#1}}
\newcommand{\grupo}[1]{\def\elgrupo{#1}}
\usepackage{graphicx}




%Vertical.
\voffset			= -15.4mm	%Off-set vertical (1in + \voffset) respecto del borde superior del papel.
\topmargin			= 0cm		%Desde el \hoffset hasta el inicio de la cabecera.
\headheight			= 5mm		%Altura de la cabecera y pie de pagina.
\headsep			= 5mm		%Distancia entre el fin de la cabecera y el inicio del cuerpo.
\topskip			= 0mm		%Distancia del inicio del cuerpo hasta la primera linea de texto.
\textheight			= 257mm 	%Altura de texto.
\footskip			= 10mm		%Distancia de las partes inferiores del texto y el pie de pagina.
\marginparpush		= 1mm		%Distancia entre los parrafos del margen (margin notes).

%Horizontal
\hoffset			= -5.4mm	%Off-set horizontal (1in + \hoffset) respecto del borde izquierdo del papel.
\oddsidemargin		= 0cm		%Desde el \voffset hasta el inicio del texto.
\textwidth			= 170mm		%Ancho de texto.
\marginparsep		= 0mm		%Separacion entre el fin del texto y el inicio del margen.
\marginparwidth		= 0mm		%Ancho del margen.
\columnsep			= 5mm		%Separacion entre columas (cuando las haya).
\columnseprule		= 0pt		%Ancho de la linea divisora de columnas (cuando las haya).

%Parrafo
\parindent			= 0mm		%Sangria al inicio del parrafo.
\parskip			= 5mm		%Distancia entre parrados.

\makeatother

\usepackage{babel}
\begin{document}
\materia{Electr\'onica II}
\trabajo{AMPLIFICADOR DE POTENCIA DE AUDIO}
\titulo{Trabajo de Laboratorio n^{\circ} 2}
\fecha{6 de Junio 2019} %Aca va la fecha. Si lo dejan asi muestra la fecha de compilacion.
\grupo{2}

%%%%%%%%%%%%%%%%%%%%%%%%%%%%%%%%%%%%%%%%%%%%%%%%%%%%%%%%%%%%%%%%

\begin{titlepage}
\begin{center}
\rule{\textwidth}{1.5pt}
\rule[0.5cm]{\textwidth}{1.5pt}
\vfill
\begin{minipage}[c]{\textwidth}
\begin{center}
\includegraphics[width=10cm]{Imagenes/LogoItba.jpg}\\
{\large{Instituto Tecnol\'ogico de Buenos Aires}}\\
{\small{\lafecha}}

\vspace{3.5cm}

\newlength{\titleWidth}
\settowidth{\titleWidth}{\Huge{\lamateria}}

{\Large{\eltrabajo}}\\
\vspace{0.7cm}
%{\normalsize{\eltitulo}}\\
\rule[2mm]{\titleWidth}{1pt}\\
{\Huge{\lamateria}}\\
\vspace*{1mm}
\rule[3mm]{\titleWidth }{1pt}

\vspace{0.5cm}

Grupo \elgrupo

\vspace{4cm}

\begin{tabular}{lcr}
\emph{Nombre}&\emph{Legajo}\\
Ariel Nowik & 58309\\
Joaqu\'in Mestanza & 58288 \\
Marcelo Regueira & 58300\\
Martina M\'aspero & 57120 \\

\end{tabular}

\end{center}
\end{minipage}
\vfill
\rule[-0.5cm]{\textwidth}{1.5pt}
\rule{\textwidth}{1.5pt}
\end{center}
\end{titlepage}
\end{document}

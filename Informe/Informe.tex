%% Preamble %%
\documentclass[paper=a4]{article}

\usepackage{float}
\usepackage{geometry}
\geometry{verbose,tmargin=2.25cm,bmargin=2cm,lmargin=2.25cm,rmargin=2cm}
\geometry{a4paper}
\usepackage{multirow}


\usepackage[T1]{fontenc}
\usepackage{fourier}
\usepackage[utf8]{inputenc}
\usepackage[spanish]{babel}				

\usepackage{amsmath,amsfonts,amsthm} % Math packages
\usepackage[pdftex]{graphicx}	

\makeatletter
%%%%%%%%%%%%%%%%%%%%%%%%%%%%%% User specified LaTeX commands.
\usepackage{fancyhdr}
\usepackage{lscape}
\pagestyle{fancy}
\lhead{Electr\'onica II 22.12}
\chead{TPL2}
\rhead{ITBA}
\renewcommand{\headrulewidth}{1pt}
\renewcommand{\footrulewidth}{1pt}

\makeatother

\usepackage{babel}
\addto\shorthandsspanish{\spanishdeactivate{~<>}}

\begin{document}

\tableofcontents
\newpage

\section{Objetivos - Par\'ametros del dise\~no}

En el presente trabajo de laboratorio se realiza el diseño y an\'alisis b\'asico del funcionamiento de una fuente regulada de tensi\'on, que cumple con las siguientes especificaciones:

\begin{center}
\begin{tabular}{|c|c|}
\hline 
Rango de tension de salida & $IO_{MAX}$\\
\hline 
\hline 
$4V \leq V_O \leq 10V$ & $1.5A$\\
\hline 
\end{tabular}
\end{center}

El diseño se implementar\'a en un PCB siguiendo determinadas consideraciones, y se realizar\'a un PCB adicional como banco de pruebas.

\section{Dise\~no del sistema}

Para implementar el dise\~no en cuesti\'on, se propone un circuito de regulaci\'on serie, el cual puede modelarse con el siguiente esquema.

PONER ESQUEMA BLOQUES

La caracter\'istica de 'serie' refiere a que el elemento de control se encuentra en serie a la carga $R_L$. En base a dicho esquema, se propone el siguiente circuito.

PONER CIRCUITO SIN CAPACITOR

La caracter\'istica de regulaci\'on se basa en un lazo de realimentaci\'on negativa entre la salida, el detector, el amplificador de error y el circuito de control. Si se supone que por un momento el valor de $V_O$ aumenta, en consecuencia el valor a la salida del detector tambi\'en aumenta. Dado que la tensi\'on provista por el generador es constante, la diferencia entre la tensi\'on a la salida del detector y el generador aumentar\'a, por lo que la tensi\'on a la salida del operacional tambi\'en. Al ocurrir esto, el transistor $T_1$ conducir\'a m\'as corriente entre colector y emisor. Dado que la corriente provista por el pre-regulador es constante (como se tratar\'a posteriormente), lo que sucede entonces es que se le quita corriente a la base del transistor $T_2$. En consecuencia, \'este conduce menos corriente, por lo que la carga $R_L$ recibe menos corriente, reestableciendo el valor de $V_O$.\par
Cada bloque por separado se trata en las subsecciones siguientes.

\subsection{Circuito de control}

El circuito de control, en este caso "serie", regula la intensidad de corriente que circula hacia la carga $R_L$ (a trav\'es de los otros bloques) de acuerdo al valor de dicha carga. El control es realizado de manera tal que el valor de $V_O$ seteado se mantenga, como se explic\'o anteriormente.

\subsubsection{Protecci\'on}

CALCULO DE LA PROTECCION, LO DE LA DERIVADA (DESPEJE EN ANEXO)

\subsubsection{Disipaci\'on de potencia}

CALCULO DISIPADOR SI NECESITA O  NO Y CUAL ELEGI AL FINAL

\subsection{Pre-regulador}

\subsection{Generador}

\subsection{Detector}

\subsection{Amplificador de error}

\subsection{Rango de carga $R_L$}

\subsection{Ganancia de lazo - Compensaci\'on}

\section{Implementaci\'on - Resultados}

CUADRO DE VALORES\\
RENDIMIENTO\\
IMPEDANCIA DE SALIDA POR MAXIMA TRANSFERENCIA DE POTENCIA BUSCANDO LA MITAD DE LA VO\\
PSRR\\



\section{Dise\~no de PCB - Consideraciones}

SARASA DE TAMECOS, PISTAS, ESPACIO PARA DISIPADOR, CONECTORES, MODELO 3D DE AMBAS PLACAS

\subsection{Placa fuente}

\subsection{Placa de banco de pruebas}

\end{document}